\documentclass[12pt, a4paper]{article}
\usepackage[T1]{fontenc}
\usepackage[portuguese]{babel}
\usepackage[left=3cm, right=2cm, top=3cm, bottom=2cm]{geometry}
\usepackage{graphicx}
\usepackage{mathtools}
\usepackage{amssymb}
\usepackage{amsthm}
\usepackage{thmtools}
\usepackage{xcolor}
\usepackage{nameref}
\usepackage{babel}
\usepackage{hyperref}
\usepackage{pxfonts}
\usepackage{calrsfs}
\usepackage{amsmath}
\usepackage{fancyhdr}

% Contador de parágrafos
\newcounter{paragraphcounter}
\setcounter{paragraphcounter}{0}

% Comando para incrementar o contador de parágrafos
\newcommand{\paragraphnum}{
	\stepcounter{paragraphcounter}
	\theparagraphcounter
}

\pagestyle{fancy}

% Configuração da numeração das equações
\makeatletter
\renewcommand{\theequation}{\theparagraphcounter.\arabic{equation}}
\@addtoreset{equation}{paragraphcounter}
\makeatother

\fancyhead[L]{Eduardo F. Rabelo - 11272697} % Cabeçalho esquerdo
\fancyhead[C]{Exercício \theparagraphcounter}     % Cabeçalho central: o capítulo
\fancyhead[R]{SFI5708 - Eletromagnetismo}      % Cabeçalho direito: número da página

\begin{document}


\textcolor{blue}{\textbf{Exercício \paragraphnum: 19/08 }
	Considere uma casca esférica não condutora uniformemente carregada. Suponha que a espessura da casca é desprezível. Calcule o campo eletrostático devido a essa distribuição de carga em um ponto da superfície da esfera. Calcule a força por unidade de área em um ponto da superfície esférica.}
\bigskip\\


O fluxo sobre uma superfície fechada pode ser descrito pela lei de Gauss:

\begin{equation}
	\begin{split}
		\oiint_S \textbf{E}\cdot d\textbf{s} & = 4\pi Q_{int} = 4\pi \sigma 4 \pi R^2 = 16\pi^2\sigma R^2 \\
		& = E4\pi r^2 = 16\pi^2\sigma R^2 \implies E = 4\pi\sigma\frac{R^2}{r^2}.
	\end{split}
\end{equation}

Sendo $R$ o raio da casca esférica e $r$ o raio da superfície Gaussiana sobre a esfera. Uma vez que o campo elétrico é paralelo à $r$ temos:

\begin{equation}
	\textbf{E} = 4\pi\sigma\frac{R^2}{r^2}\hat{r}
	\label{eq1:campo}
\end{equation}

Como queremos calcular o campo elétrico sujeito a um elemento de área na superfície da esfera temos que considerar o campo médio entre os campos. Nesse caso, como não temos cargas no interior da esfera podemos considerar a contribuição interna do campo elétrico como nula, logo

\begin{equation}
	\textbf{E}_{med} = \left( \frac{\textbf{E}_{int} + \textbf{E}_{ext}}{2}\right) \implies \textbf{E}_{med} = \frac{\textbf{E}_{ext}}{2}; \qquad \textbf{E}_{ext} = \textbf{E}
	\label{eq1:campo_medio}
\end{equation}

Finalmente, para calcularmos a força gerada pelo campo elétrico podemos considera $\textbf{F} = q\textbf{E}$, no caso para considerarmos um elemento de área $dA$ temos
`
\begin{equation}
	\frac{dq}{dA} = \sigma \implies dq = \sigma dA
\end{equation}

\begin{equation}
	d \textbf{F} = \textbf{E}_{med} dA = \textbf{E}_{med} \sigma dA 
	\label{eq1:elemento_forca}
\end{equation}

substituindo \ref{eq1:campo} e \ref{eq1:campo_medio} em \ref{eq1:elemento_forca} temos

\begin{equation}
	\frac{d\textbf{F}}{dA} = \frac{1}{2} \sigma 4\pi \frac{R^2}{r^2} \sigma \hat{r} = 2\pi \sigma^2 \frac{R^2}{r^2} \hat{r}
\end{equation}

para que tomemos os resultados na superfície da esfera consideramos o limite em que $r \rightarrow R$. Portanto:

\begin{equation}
	\frac{d\textbf{F}}{dA} = 2\pi \sigma^2 \hat{r}
\end{equation}.

\newpage
			
\textcolor{blue}{
	\textbf{Exercício \paragraphnum: 21/08 }
	\begin{itemize}
		\item Demonstre o teorema de Helmholtz, isto é, demonstre as Eqs. (6.47), (6.49) e (6.50) da segunda edição do livro de J. D. Jackson ou as Eqs. (6.25), (6.27) e (6.28), da terceira.
		\item Refaça detalhadamente os cálculos para obter a função de Green para a equação de onda.
	\end{itemize}
}


Queremos encontrar a solução para a equação de onda resultante das transformações por Gauge de Lorentz no qual queremos encontrar a solução particular que satisfaça

\begin{equation}
	\nabla ^2 \Psi (\textbf{r},t) -\frac{1}{c^2}\frac{\partial^2}{\partial t^2} \Psi = 
	f(\textbf{r},t) 
	\label{eq2:onda}
\end{equation}

Portanto, queremos encontrar a função de Green que satisfaz a equação \ref{eq2:onda}. Das propriedades da função de Green temos

\begin{equation}
	\begin{split}
	&\nabla^2 G(\textbf{r}, t, \textbf{r}', t')  -\frac{1}{c^2}\frac{\partial^2}{\partial t^2}G(\textbf{r}, t, \textbf{r}', t') = 
	\delta^{(3)}(\textbf{r}-\textbf{r}') \delta (t-t') 
	\\
	&\left( \nabla^2 -\frac{1}{c^2}\frac{\partial^2}{\partial t^2}\right)  G(\textbf{r}, t, \textbf{r}', t') =
	\delta^{(3)}(\textbf{r}-\textbf{r}') \delta (t-t')
	\end{split}
	\label{eq2:func_green}
\end{equation}

no qual possibilitamos a construção da solução particular $\Psi_P$

\begin{equation}
	\Psi (\textbf{r}, t)_P= 
	\int d^3 r \int_{-\infty}^{+\infty} dt G(\textbf{r}, t, \textbf{r}') f(\textbf{r}', t')
	\label{eq2:psi}
\end{equation}

Para facilitar nossas contas vamos passar do espaço temporal para o de frequências por meio de transformada de Fourier sobre $t$

\begin{equation}
	G(\textbf{r},t, \textbf{r}', t') = 
	\int_{-\infty}^{+\infty} d\omega e^{i\omega t} g(\textbf{r},\omega, \textbf{r}', t')
	\label{eq2:fourier}
\end{equation}

Substituindo \ref{eq2:fourier} em \ref{eq2:func_green} podemos calcular explicitamente as derivadas temporais e passar de operadores de derivada temporal para termos geométricos da forma $k_0 = \frac{\omega}{c}$.

\begin{equation}
	\begin{split}
	& \left( \nabla^2-\frac{1}{c^2}\frac{\partial^2}{\partial t^2}\right) 
	\int_{-\infty}^{+\infty} d\omega e^{i\omega t} g(\textbf{r},\omega, \textbf{r}', t')\\
	& = \left( \nabla^2 + \frac{\omega^2}{c^2}\right)
	\int_{-\infty}^{+\infty} d\omega e^{i\omega t} g(\textbf{r},\omega, \textbf{r}', t')\\
	& = \left( \nabla^2 + k_0^2\right) \int_{-\infty}^{+\infty} d\omega e^{i\omega t} g(\textbf{r},\omega, \textbf{r}', t')
	\end{split}
\end{equation}

Olhando para o lado direito da equação \ref{eq2:func_green} quando tomamos a transformada de Fourier temos que levar em consideração que a delta de Dirac passa a ser escrita também na sua forma transformada.

\begin{equation}
	\delta(t-t') = \frac{1}{2\pi}\int_{-\infty}^{+\infty} d\omega e^{-i\omega (t-t')}
	\label{eq2:delta}
\end{equation}

Logo, chegamos que

\begin{equation}
	\left( \nabla^2 + k_0^2\right) \int_{-\infty}^{+\infty} d\omega e^{i\omega t} g(\textbf{r},\omega, \textbf{r}', t') = \delta(\textbf{r}-\textbf{r}')\frac{1}{2\pi}\int_{-\infty}^{+\infty} d\omega e^{-i\omega (t-t')}.
\end{equation}

Simplificar essa equação pode nos ajudar a visualizar o problema que vamos trabalhar nesse momento, logo subtraindo o lado direito da equação em ambos os lados e colocando tudo dentro do mesmo integrando com a exponencial de $t$ em evidência

\begin{equation}
	\int_{-\infty}^{+\infty} d\omega \quad 
	e^{-i\omega t}
	\left[ \left( \nabla^2 + k_0^2\right) g(\textbf{r},\omega, \textbf{r}', t') 
	- \frac{e^{i\omega t'}}{2\pi}
	\delta^{(3)}(\textbf{r}-\textbf{r}')\right] = 0 
\end{equation}

Notemos que essa equação será verdadeira no caso dos termos internos se anularem uma vez que a exponencial dependente de $t$ nunca será nula.

\begin{equation}
	\left( \nabla^2 + k_0^2\right) g(\textbf{r},\omega, \textbf{r}', t') 
    = \frac{e^{i\omega t'}}{2\pi}\delta^{(3)}(\textbf{r}-\textbf{r}')
    \label{eq2:simplificada}
\end{equation}

Nesse ponto, o que queremos é encontrar o valor da função de Green ($G$) a partir de $g$. Nesse caso, temos que levar em consideração a existência de uma singularidade no denominador ao isolarmos $g$ uma vez que a solução do sistema divergiria no eixo dos reais. Para resolver esse problema vamos induzir um polo no semi-espaço dos complexos por meio de uma aproximação $g_\eta$ na qual

\begin{equation*}
	g_{\eta} = g_{\eta}(\textbf{r}, \omega, \textbf{r}', t')
\end{equation*}

que resultaria na correção infinitesimal em \ref{eq2:simplificada} correspondente à

\begin{equation}
	\left( \nabla^2 + (k_0+i\eta)^2\right) g_{\eta}(\textbf{r},\omega, \textbf{r}', t') 
	= \frac{e^{i\omega t'}}{2\pi}\delta^{(3)}(\textbf{r}-\textbf{r}')
\end{equation}

Realizando agora a transformada de Fourier da posição para eliminar o operador de derivada temos

\begin{equation}
	g_{\eta}(\textbf{r}, \omega, \textbf{r}', t') =
	\int d^3k \quad e^{i\textbf{k}\cdot\textbf{r}}\bar{g}_{\eta}(\textbf{k}, \omega, \textbf{r}', t')
\end{equation}

que nos dá

\begin{equation}
	\int d^3k \quad e^{i\textbf{k}\cdot\textbf{r}} \left( k^2 + (k_0+i\eta)^2\right) \bar{g_\eta}(\textbf{k},\omega, \textbf{r}', t') 
	= \frac{e^{i\omega t'}}{2\pi}\delta^{(3)}(\textbf{r}-\textbf{r}')
\end{equation}

uma vez que

\begin{equation}
	\begin{split}
	 & \nabla^2 g_{\eta}(\textbf{k}, \omega, \textbf{r}', t')
	= \nabla^2 e^{i\textbf{k}\cdot\textbf{r}}\bar{g}_{\eta}(\textbf{k}, \omega, \textbf{r}', t')\\
	= &(-\textbf{k}\cdot\textbf{k}) e^{i\textbf{k}\cdot\textbf{r}}\bar{g}_{\eta}(\textbf{k}, \omega, \textbf{r}', t')
	= -k^2 e^{i\textbf{k}\cdot\textbf{r}}\bar{g}_{\eta}(\textbf{k}, \omega, \textbf{r}', t')
	\end{split}
\end{equation}

Finalmente podemos encontrar uma equação $\bar{g}_\eta$ que pode está a uma integral de nos dar o resultado para a função de Green desejada desde o início.

\begin{equation}
	\left( \nabla^2 + (k_0+i\eta)^2\right) g_{\eta}(\textbf{r},\omega, \textbf{r}', t') 
	= \frac{e^{i\omega t'}}{2\pi}\delta^{(3)}(\textbf{r}-\textbf{r}')
\end{equation}

\begin{equation}
	\int d^3 k \quad \left( -k^2 + (k_0+i\eta)^2\right) \bar{g}_{\eta}(\textbf{r},\omega, \textbf{r}', t') 
	= \frac{e^{i\omega t'}}{2\pi}\delta^{(3)}(\textbf{r}-\textbf{r}')
\end{equation}

Substituindo a delta explicitamente, assim como na equação \ref{eq2:delta}

\begin{equation}
	\int d^3 k \quad \left( -k^2 + (k_0+i\eta)^2\right) e^{i\textbf{k}\cdot\textbf{r}}\bar{g}_{\eta}(\textbf{r},\omega, \textbf{r}', t') 
	= \frac{e^{i\omega t'}}{2\pi}\frac{1}{(2\pi)^3} \int d^3k \quad e^{i\textbf{k}\cdot(\textbf{r}-\textbf{r}')}
\end{equation}

Igualando os integrandos, temos

\begin{equation}
	\left( -k^2 + (k_0+i\eta)^2\right) e^{i\textbf{k}\cdot\textbf{r}} \bar{g}_{\eta}(\textbf{r},\omega, \textbf{r}', t') = \frac{e^{i\omega t'}}{(2\pi)^4}e^{i\textbf{k}}\cdot(\textbf{r}-\textbf{r}')
\end{equation}

logo,

\begin{equation}
	 \bar{g}_{\eta}(\textbf{r},\omega, \textbf{r}', t') = \frac{e^{i\omega t'}}{(2\pi)^4}\frac{e^{i\textbf{k}\cdot(\textbf{r}-\textbf{r}')}}{e^{i\textbf{k}\cdot\textbf{r}} \left( -k^2 + (k_0+i\eta)^2\right)} = \frac{e^{i\omega t'-i\textbf{k}\cdot\textbf{r}'}}{(2\pi)^4\left( -k^2 + (k_0+i\eta)^2\right)}
\end{equation}

Substituindo $\bar{g}_\eta$ em $g_{\eta}$, em seguida em $g_{\eta}$ em $G$ conseguiremos encontrar o valor para $G_{\eta}$

\begin{equation}
	\begin{split}
	g_{\eta}(\textbf{r}, \omega, \textbf{r}', t') = &
	\int d^3k \quad e^{i\textbf{k}\cdot\textbf{r}}\bar{g}_{\eta}(\textbf{k}, \omega, \textbf{r}', t') \\
	= & \int d^3k \quad e^{i\textbf{k}\cdot\textbf{r}} \frac{e^{i\omega t'-i\textbf{k}\cdot\textbf{r}'}}{(2\pi)^4\left( -k^2 + (k_0+i\eta)^2\right)}
	\end{split}
\end{equation}

finalmente,

\begin{equation}
	\begin{split}
	G_{\eta}(\textbf{r},t, \textbf{r}', t') = & 
	\int_{-\infty}^{+\infty} d\omega e^{-i\omega t} g(\textbf{r},\omega, \textbf{r}', t') \\
	= & \int_{-\infty}^{+\infty} d\omega e^{i\omega t} \int d^3k \quad e^{i\textbf{k}\cdot\textbf{r}} \frac{e^{i\omega t'-i\textbf{k}\cdot\textbf{r}'}}{(2\pi)^4\left( -k^2 + (k_0+i\eta)^2\right)}
	\end{split}.
\end{equation}

Para obtermos os resultados desejados para a solução, a partir de agora basta realizar as integrações em $d^3k$ e $d\omega$, uma vez que para facilitar podemos escrever a equação da seguinte forma:

\begin{equation}
	G_{\eta}(\textbf{r},t, \textbf{r}', t') = G_{\eta}(\textbf{r}-\textbf{r}',t-t') = \int_{-\infty}^{+\infty} d\omega \int d^3k \quad  \frac{e^{i\textbf{k}\cdot(\textbf{r}-\textbf{r}')-i\omega (t-t')}}{(2\pi)^4\left( -k^2 + (k_0+i\eta)^2\right)}
\end{equation}

ou então,

\begin{equation}
	G_{\eta}(\textbf{r},t) = G_{\eta}(\textbf{r}-\textbf{r}',t-t') \int_{-\infty}^{+\infty} d\omega \int d^3k \quad  \frac{e^{i\textbf{k}\cdot\textbf{r}-i\omega t}}{(2\pi)^4\left( -k^2 + (k_0+i\eta)^2\right)}
\end{equation}

uma vez que $G_{\eta}(\textbf{r}-\textbf{r}',t-t') \quad \exists \quad \forall \quad \textbf{r}-\textbf{r}' \text{ e } t-t'.$ 

Aplicando a transformação de coordenadas polares para coordenadas esféricas temos

\begin{equation}
	\begin{split}
	& \int_V d^3k \quad  \frac{e^{i\textbf{k}\cdot\textbf{r}-i\omega t}}{(2\pi)^4\left( -k^2 + (k_0+i\eta)^2\right)}\\
	 = &  \int_0^{2\pi}d\phi \int_{1}^{-1} e^{i\textbf{k}\cdot\textbf{r}} d\cos\theta \int_{0}^{R} \frac{k^2}{(2\pi)^4\left( -k^2 + (k_0+i\eta)^2\right)}dk \\
	 = &  (2\pi) \int_{1}^{-1} e^{ikr\cos\theta} d\cos\theta \int_{0}^{R} \frac{k^2}{(2\pi)^4\left( -k^2 + (k_0+i\eta)^2\right)}dk\\
	 = & \int_{0}^{R} \frac{2\pi \left(-e^{ikr}+e^{-ikr}\right)}{ikr} \frac{k^2}{(2\pi)^4\left( -k^2 + (k_0+i\eta)^2\right)}dk\\
	 = & \int_{-R}^{R} \frac{ke^{ikr}}{(2\pi)^3 ir\left( -k^2 + (k_0+i\eta)^2\right)}dk
	\end{split}
\end{equation}

No contexto de integrais complexas podemos desconsiderar o caminho que passa fora do eixo dos reais quando fechamos a superfície ao redor dos polos. Isso pode ser facilmente demonstrado quando olhamos para

\begin{equation*}
	\begin{split}
	&z = Re^{i\theta} = Rcos\theta + iRsin\theta \\
	&e^{irz} = e^{ir(Rcos\theta + iRsin\theta)} = e^{irRcos\theta - rRsin\theta}\\
	&e^{irz} = e^{irRcos\theta}e^{-rRsin\theta}
	\end{split}
\end{equation*}

Quando tomamos o limite de R tendendo a $\infty$ temos uma contribuição nula da parte complexa, e fazendo apenas a integral na reta real com $\eta$ no limite de 0 por dentro da curva. Para finalizar, basta fazermos a integração utilizando o teorema dos resíduos. Vale lembrar que,

\begin{equation}
	\oint dz \quad \frac{ze^{izr}}{z^2-(k_0+i\eta)^2} = 2\pi i \text{Res}(z_\pm)
\end{equation}

com

\begin{equation}
	z^2-(k_0+i\eta)^2 = (z-z_+)(z-z_-) \impliedby \begin{cases} 
		z_+ =& k_0 + i\eta \\
		z_- =& -k_0 - i\eta 
	\end{cases}
\end{equation}

Calculando os resíduos em $z_+$ temos,

\begin{equation}
	\text{Res}(z_+) = \lim_{\eta\rightarrow 0^+} \frac{k_+ e^{iz_+r}}{z_+-z_-} = \frac{k_0 e^{ik_0r}}{k_0-(-k_0)} = \frac{e^{ik_0r}}{2}
\end{equation}

já em $z_-$ temos:
\begin{equation}
	\text{Res}(z_-) = \lim_{\eta\rightarrow 0^-} \frac{k_- e^{iz_-r}}{z_--z_+} = \frac{-k_0 e^{-ik_0r}}{-k_0-k_0} = \frac{e^{-ik_0r}}{2}
\end{equation}


Por fim, podemos substituir os resultados para os resíduos na função de Green, lembrando de considerar a integral sobre $\omega$ e a constante $2\pi i$ enunciada anteriormente no teorema.

\begin{equation}
	\begin{split}
		G_{\pm}(\mathbf{r},t) &= \int_{-\infty}^{+\infty} d\omega \, e^{-i\omega t} \int_{-R}^{R} \frac{ke^{ikr}}{(2\pi)^3 ir\left( -k^2 + (k_0+i\eta)^2\right)} \, dk \\
		&= \int_{-\infty}^{+\infty} d\omega \frac{e^{-i\omega t}}{(2\pi)^3 ir} \int_{-\infty}^{\infty} \frac{ke^{ikr}}{\left( -k^2 + (k_0+i\eta)^2\right)} \, dk \\
		&= \int_{-\infty}^{+\infty} d\omega \frac{e^{-i\omega t}}{(2\pi)^3 ir} \cdot 2\pi i \, \text{Res}(z_\pm) \\
		&= \int_{-\infty}^{+\infty} d\omega \frac{e^{-i\omega t}}{(2\pi)^3 ir} \cdot 2\pi i \cdot \frac{e^{\pm ik_0 r}}{2} \\
		&= \int_{-\infty}^{+\infty} d\omega \frac{e^{-i\omega t}}{(2\pi)^3 ir} \cdot \pi i \, e^{\pm i \frac{\omega c}{r}} \\
		&= -\frac{1}{(2\pi)^3} \frac{\pi}{r} \int_{-\infty}^{+\infty} d\omega \, e^{-i\omega t} \\
		&= -\frac{1}{(2\pi)^3} \frac{\pi}{r} \delta \left( t \mp \frac{r}{c} \right) \\
	\end{split}
\end{equation}

Substituindo em \ref{eq2:psi} e retornando os tempos e posições $t-t'$ e $\textbf{r}-\textbf{r}'$ considerando que definimos essa possibilidade anteriormente, temos:

\begin{equation}
	\Psi (\textbf{r}, t)_{\pm} = 
	-\int d^3 r' \int_{-\infty}^{+\infty} dt' \frac{1}{4\pi|\textbf{r}-\textbf{r}'|}\delta\left( t-t'\mp\frac{|\textbf{r}-\textbf{r}'|}{c}\right) f(\textbf{r}', t')
\end{equation}

Resolvendo para t,

\begin{equation}
	\Psi (\textbf{r}, t)_{\pm} = 
	-\int d^3 r'\frac{f(\textbf{r}', t\mp\frac{|\textbf{r}-\textbf{r}'|}{c})}{4\pi|\textbf{r}-\textbf{r}'|}
\end{equation}

Nesse contexto, como temos duas soluções, utilizaremos a solução de tempo retardado, que é definido pelas consequências serem geradas após a interação. A outra escolha é chamado de tempo adiantado, mas pode não ser muito interessante no momento uma vez que estaríamos considerando, por exemplo, uma medida antes de realizar o experimento.

\begin{equation}
	\Psi (\textbf{r}, t)_{+} = 
	-\int d^3 r'\frac{f(\textbf{r}', t -\frac{|\textbf{r}-\textbf{r}'|}{c})}{4\pi|\textbf{r}-\textbf{r}'|}
\end{equation}

Para finalizar, podemos substituir os resultados encontrados nos pares ordenados das fontes discutidos em sala de aula, desta forma encontrando uma solução para evolução temporal com relação as fontes

\begin{equation*}
	\phi \rightarrow f = -4\pi\rho(\textbf{r},t)
\end{equation*}

\begin{equation*}
	\textbf{A} \rightarrow f = \frac{-4\pi}{c}\textbf{J}
\end{equation*}

logo,

\begin{equation}
	\phi (\textbf{r}, t) = 
	\int d^3 r'\frac{\rho(\textbf{r}', t -\frac{|\textbf{r}-\textbf{r}'|}{c})}{|\textbf{r}-\textbf{r}'|}
	\label{eq2:potencialescalar}
\end{equation}

\begin{equation}
	\textbf{A} (\textbf{r}, t) = 
	\frac{1}{c}\int d^3 r'\frac{\textbf{J}(\textbf{r}', t -\frac{|\textbf{r}-\textbf{r}'|}{c})}{|\textbf{r}-\textbf{r}'|}	
\end{equation}

O vetor potencial satisfaz a equação de onda não homogênea

\begin{equation}
	\mathbf{\nabla} ^2 -\frac{1}{c^2}\frac{\partial^2\mathbf{A}}{\partial t^2} = -\frac{4\pi}{c}\textbf{J} + \frac{1}{c}\mathbf{\nabla}\frac{\partial \phi}{\partial t}
\end{equation}

por isso, desde que estejamos envolvendo um operador gradiente temos um termo irrotacional. Por isso, deve cancelar com outra componente correspondente da densidade de corrente satisfazendo então,

\begin{equation}
	\textbf{J} = \textbf{J}_l + \textbf{J}_t
\end{equation}

uma vez que $\nabla \times \textbf{J}_l = 0$ e $\nabla \cdot \textbf{J}_t = 0$







\newpage

\textcolor{blue}{\textbf{Exercício \paragraphnum: 28/08 }
	Refaça detalhadamente os cálculos sobre a conservação de momentum linear em eletromagnetismo do PDF da aula de hoje.}
\bigskip\\


Do que já sabemos da mecânica clássica, a variação do momento de um sistema é descrito pela força externa que atua sobre o mesmo.

\begin{equation}
	\textbf{F} = \frac{d}{dt} \textbf{P}
\end{equation}

Queremos nessa etapa, analisar de alguma forma a conservação de momento de um sistema eletromagnético e, para isso, vamos considerar a variação de momentum linear da matéria carregada ($\frac{d}{dt} \textbf{P}_m$) igual a força de Lorentz por unidade de área expressa sobre um volume com distribuição de carga $\rho$

\begin{equation}
	d \textbf{F} = d^3 r \left( \rho \textbf{E} + \frac{\rho}{c} \textbf{v} \times \textbf{B} \right)
\end{equation}

e força total $F_V$

\begin{equation}
	\begin{split}
		F_V & = \int_V d^3 r \left( \rho \textbf{E} + \frac{\rho}{c} \textbf{v} \times \textbf{B} \right) \\
		& = \int_V d^3 r \left( \rho \textbf{E} + \frac{1}{c} \textbf{J} \times \textbf{B} \right) = \frac{d}{dt} \textbf{P}_m
		\label{eq3:variacaodemomento}
	\end{split}
\end{equation}

Entretanto, não demonstraremos isso em função das fontes, e sim trabalharemos nessa demonstração majoritariamente através das equações de Maxwell. Para isso, vamos apenas fazer um lembrete de todas elas:
\\

\begin{tcolorbox}[title=Equações de Maxwell, colframe=blue, colback=blue!5!white, coltitle=white, fonttitle=\bfseries]
	
	\textbf{Lei de Gauss}
	
	\begin{equation}
		\mathbf{\nabla} \cdot \textbf{E}  = 4 \pi \rho
		\label{eq3:leidegauss}
	\end{equation}
	
	\textbf{Inexistência de monopolos magnéticos}
	\begin{equation}
		\mathbf{\nabla} \cdot \textbf{B} = 0
		\label{eq3:monopolos}
	\end{equation}
	
	\textbf{Lei de indução de Faraday}
	\begin{equation}
		\mathbf{\nabla} \times \textbf{E} = - \frac{\partial}{\partial t} \textbf{B}
		\label{eq3:leidefaraday}
	\end{equation}
	
	\textbf{Lei de Ampére-Maxwell}
	\begin{equation}
		\mathbf{\nabla} \times \textbf{B} = \frac{4\pi}{c} \textbf{J} +  \frac{1}{c}\frac{\partial}{\partial t}\textbf{E}
		\label{leideampere}
	\end{equation}
	
\end{tcolorbox}


Utilizando a lei de Gauss e a lei de Ampére-Maxwell para descrever as fontes temos

\begin{equation}
	\begin{split}
		\rho & = \frac{1}{4\pi}\mathbf{\nabla} \cdot \textbf{E} \\
		\textbf{J} & = \frac{c}{4\pi} \mathbf{\nabla} \times \textbf{B} - \frac{1}{4\pi}\frac{\partial}{\partial t}\textbf{E}
	\end{split}
\end{equation}

podemos então substituir em \ref{eq3:variacaodemomento}.

\begin{equation}
	\begin{split}
		\frac{d}{dt} \textbf{P}_m & = \int_V d^3 r \left( \rho \textbf{E} + \frac{1}{c} \textbf{J} \times \textbf{B} \right) \\
		& = \int_V d^3 r \left( \frac{1}{4\pi}\left( \mathbf{\nabla} \cdot \textbf{E}\right)  \textbf{E} + \frac{1}{4\pi} \mathbf{\nabla} \times \textbf{B} - \frac{1}{4\pi c}\frac{\partial}{\partial t}\textbf{E} \times \textbf{B} \right)
	\end{split}
	\label{eq3:semfontes}
\end{equation}

Podemos facilmente trabalhar com a Lei de Faraday (\ref{eq3:leidefaraday}) na igualdade acima, a regra do produto nos dá que

\begin{equation}
	\begin{split}
		\frac{\partial}{\partial t}\left( \textbf{E} \times \textbf{B}\right) = \frac{\partial}{\partial t}\textbf{E} \times \textbf{B} + \textbf{E} \times \frac{\partial}{\partial t}\textbf{B} \implies \\
		\frac{\partial}{\partial t}\textbf{E} \times \textbf{B} = \frac{\partial}{\partial t}\left( \textbf{E} \times \textbf{B}\right) - \textbf{E} \times \frac{\partial}{\partial t}\textbf{B}.
	\end{split}
\end{equation}

Substituindo em \ref{eq3:semfontes} e em seguida substituindo a derivada temporal do campo magnético pela Lei de Faraday:

\begin{equation}
	\begin{split}
		& \int_V d^3 r \left[ \frac{1}{4\pi}\left( \mathbf{\nabla} \cdot \textbf{E}\right)  \textbf{E} + \frac{1}{4\pi} \mathbf{\nabla} \times \textbf{B} - \frac{1}{4\pi c} \left(\frac{\partial}{\partial t}\left( \textbf{E} \times \textbf{B}\right) - \textbf{E} \times \frac{\partial}{\partial t}\textbf{B}\right) \right]\\
		= & \int_V d^3 r \left[ \frac{1}{4\pi}\left( \mathbf{\nabla} \cdot \textbf{E}\right)  \textbf{E} + \frac{1}{4\pi} \mathbf{\nabla} \times \textbf{B} - \frac{1}{4\pi c} \left(\frac{\partial}{\partial t}\left( \textbf{E} \times \textbf{B}\right) + c\textbf{E} \times (\mathbf{\nabla} \times \textbf{B})\right)\right] \\
		= & \int_V d^3 r \left[ \frac{1}{4\pi}\left( \mathbf{\nabla} \cdot \textbf{E}\right)  \textbf{E} + \frac{1}{4\pi} \mathbf{\nabla} \times \textbf{B} - \frac{1}{4\pi c} \frac{\partial}{\partial t}\left( \textbf{E} \times \textbf{B}\right) - \frac{1}{4\pi }\textbf{E} \times (\mathbf{\nabla} \times \textbf{B})\right]
	\end{split}
\end{equation}
	
\newpage

\textcolor{blue}{\textbf{Exercício \paragraphnum: 02/09 }
	Considere duas cargas pontuais idênticas separadas por uma distância finita no vácuo. Faça a integral de superfície do tensor dos estresses de Maxwell sobre o plano dos pontos equidistantes às duas cargas. Deduza, assim, a força de Coulomb entre as duas cargas.}
\bigskip\\
\newpage
\end{document}