\textcolor{blue}{\textbf{Exercício \paragraphnum: 28/08 }
	Refaça detalhadamente os cálculos sobre a conservação de momentum linear em eletromagnetismo do PDF da aula de hoje.}
\bigskip\\


Do que já sabemos da mecânica clássica, a variação do momento de um sistema é descrito pela força externa que atua sobre o mesmo.

\begin{equation}
	\textbf{F} = \frac{d}{dt} \textbf{P}
\end{equation}

Queremos nessa etapa, analisar de alguma forma a conservação de momento de um sistema eletromagnético e, para isso, vamos considerar a variação de momentum linear da matéria carregada ($\frac{d}{dt} \textbf{P}_m$) igual a força de Lorentz por unidade de área expressa sobre um volume com distribuição de carga $\rho$

\begin{equation}
	d \textbf{F} = d^3 r \left( \rho \textbf{E} + \frac{\rho}{c} \textbf{v} \times \textbf{B} \right)
\end{equation}

e força total $F_V$

\begin{equation}
	\begin{split}
		F_V & = \int_V d^3 r \left( \rho \textbf{E} + \frac{\rho}{c} \textbf{v} \times \textbf{B} \right) \\
		& = \int_V d^3 r \left( \rho \textbf{E} + \frac{1}{c} \textbf{J} \times \textbf{B} \right) = \frac{d}{dt} \textbf{P}_m
		\label{eq3:variacaodemomento}
	\end{split}
\end{equation}

Entretanto, não demonstraremos isso em função das fontes, e sim trabalharemos nessa demonstração majoritariamente através das equações de Maxwell. Para isso, vamos apenas fazer um lembrete de todas elas:
\\

\begin{tcolorbox}[title=Equações de Maxwell, colframe=blue, colback=blue!5!white, coltitle=white, fonttitle=\bfseries]
	
	\textbf{Lei de Gauss}
	
	\begin{equation}
		\mathbf{\nabla} \cdot \textbf{E}  = 4 \pi \rho
		\label{eq3:leidegauss}
	\end{equation}
	
	\textbf{Inexistência de monopolos magnéticos}
	\begin{equation}
		\mathbf{\nabla} \cdot \textbf{B} = 0
		\label{eq3:monopolos}
	\end{equation}
	
	\textbf{Lei de indução de Faraday}
	\begin{equation}
		\mathbf{\nabla} \times \textbf{E} = - \frac{\partial}{\partial t} \textbf{B}
		\label{eq3:leidefaraday}
	\end{equation}
	
	\textbf{Lei de Ampére-Maxwell}
	\begin{equation}
		\mathbf{\nabla} \times \textbf{B} = \frac{4\pi}{c} \textbf{J} +  \frac{1}{c}\frac{\partial}{\partial t}\textbf{E}
		\label{leideampere}
	\end{equation}
	
\end{tcolorbox}


Utilizando a lei de Gauss e a lei de Ampére-Maxwell para descrever as fontes temos

\begin{equation}
	\begin{split}
		\rho & = \frac{1}{4\pi}\mathbf{\nabla} \cdot \textbf{E} \\
		\textbf{J} & = \frac{c}{4\pi} \mathbf{\nabla} \times \textbf{B} - \frac{1}{4\pi}\frac{\partial}{\partial t}\textbf{E}
	\end{split}
\end{equation}

podemos então substituir em \ref{eq3:variacaodemomento}.

\begin{equation}
	\begin{split}
		\frac{d}{dt} \textbf{P}_m & = \int_V d^3 r \left( \rho \textbf{E} + \frac{1}{c} \textbf{J} \times \textbf{B} \right) \\
		& = \int_V d^3 r \left( \frac{1}{4\pi}\left( \mathbf{\nabla} \cdot \textbf{E}\right)  \textbf{E} + \frac{1}{4\pi} \left( \mathbf{\nabla} \times \textbf{B}\right) \times \textbf{B} - \frac{1}{4\pi c}\frac{\partial}{\partial t}\textbf{E} \times \textbf{B} \right)
	\end{split}
	\label{eq3:semfontes}
\end{equation}

Podemos facilmente trabalhar com a Lei de Faraday (\ref{eq3:leidefaraday}) na igualdade acima, a regra do produto nos dá que

\begin{equation}
	\begin{split}
		\frac{\partial}{\partial t}\left( \textbf{E} \times \textbf{B}\right) = \frac{\partial}{\partial t}\textbf{E} \times \textbf{B} + \textbf{E} \times \frac{\partial}{\partial t}\textbf{B} \implies \\
		\frac{\partial}{\partial t}\textbf{E} \times \textbf{B} = \frac{\partial}{\partial t}\left( \textbf{E} \times \textbf{B}\right) - \textbf{E} \times \frac{\partial}{\partial t}\textbf{B}.
	\end{split}
\end{equation}

Substituindo em \ref{eq3:semfontes} e em seguida substituindo a derivada temporal do campo magnético pela Lei de Faraday:

\begin{equation}
	\begin{split}
		& \int_V d^3 r \left[ \frac{1}{4\pi}\left( \mathbf{\nabla} \cdot \textbf{E}\right)  \textbf{E} + \frac{1}{4\pi} \left( \mathbf{\nabla} \times \textbf{B}\right) \times \textbf{B} - \frac{1}{4\pi c} \left(\frac{\partial}{\partial t}\left( \textbf{E} \times \textbf{B}\right) - \textbf{E} \times \frac{\partial}{\partial t}\textbf{B}\right) \right]\\
		= & \int_V d^3 r \left[ \frac{1}{4\pi}\left( \mathbf{\nabla} \cdot \textbf{E}\right)  \textbf{E} + \frac{1}{4\pi} \left( \mathbf{\nabla} \times \textbf{B}\right) \times \textbf{B} - \frac{1}{4\pi c} \left(\frac{\partial}{\partial t}\left( \textbf{E} \times \textbf{B}\right) + c\textbf{E} \times (\mathbf{\nabla} \times \textbf{E})\right)\right] \\
		= & \int_V d^3 r \left[ \frac{1}{4\pi}\left( \mathbf{\nabla} \cdot \textbf{E}\right)  \textbf{E} - \frac{1}{4\pi}  \textbf{B} \times \left( \mathbf{\nabla} \times \textbf{B}\right) - \frac{1}{4\pi c} \frac{\partial}{\partial t}\left( \textbf{E} \times \textbf{B}\right) - \frac{1}{4\pi }\textbf{E} \times (\mathbf{\nabla} \times \textbf{E})\right]
	\end{split}
\end{equation}

uma vez que não existem monopolos magnéticos, isto é, $\nabla \cdot B = 0$ podemos acrescentar um termo para fins de simetria da equação. Por isso ficamos com,

\begin{equation}
	\begin{split}
		& \frac{d}{dt} \textbf{P}_m + \int_V d^3 r \frac{1}{4\pi c} \frac{\partial}{\partial t}\left( \textbf{E} \times \textbf{B}\right) = \\
		\qquad & =	\int_V d^3 r \left[ \frac{1}{4\pi}\left( \mathbf{\nabla} \cdot \textbf{E}\right)  \textbf{E} + \frac{1}{4\pi}\left( \mathbf{\nabla} \cdot \textbf{B}\right) \textbf{B}- \frac{1}{4\pi}  \textbf{B} \times \left( \mathbf{\nabla} \times \textbf{B}\right) - \frac{1}{4\pi }\textbf{E} \times (\mathbf{\nabla} \times \textbf{E})\right]
	\end{split}
\end{equation}

Os termos escalares na realidades são díades na forma $E_{i}\partial_{j}E_k$. Utilizando as convenções de Einstein, podemos olhar para os termos rotacionais acima onde temos,

\begin{equation*}
	\begin{split}
	\left[ \textbf{E} \times \left( \nabla \times \textbf{E}\right) \right]_k & = \epsilon_{kmn}E_m\left( \nabla \times \textbf{E}\right)_n  \\ 
	& = \epsilon_{kmn}E_m\epsilon_{njr}\partial_jE_r \\
	& = \epsilon_{kmn}\epsilon_{njr}E_mE_r\\
	& = \left(\delta_{kj}\delta_{mr}-\delta_{kr}\delta_{mj}\right) E_mE_r \\
	& = \delta_{kj}E_m\partial_j E_m - E_j\partial_j E_k \\
	& = \frac{1}{2}\delta_{kj}\textbf{E}^2
	\end{split}
\end{equation*}

que possui uma estrutura muito similar à díade falada anteriormente. Para o campo elétrico, se considerarmos o lema de Gauss, temos:

\begin{equation}
	\begin{split}
	& \int_V d^3 r \left[ \frac{1}{4\pi}\left( \mathbf{\nabla} \cdot \textbf{E}\right)  \textbf{E} - \frac{1}{4\pi }\textbf{E} \times (\mathbf{\nabla} \times \textbf{E}) \right] \\
	= & \int_V d^3 r \frac{1}{4\pi } \left[\partial_k\textbf{E}E_k - \frac{1}{2}\delta_{kj}\textbf{E}^2 \right] \\
	= & \frac{1}{4\pi }\oint_{S(V)} da\quad n_k\textbf{E}E_k -\frac{1}{4\pi } \oint_{S(V)} da\quad \frac{1}{2}n_k E_k^2\\
	= & \frac{1}{4\pi }\oint_{S(V)} da\quad \left[ (\hat{n}\cdot \textbf{E})\textbf{E} - \frac{1}{2}\hat{n} (\textbf{E}\cdot\textbf{E})\right]
	\end{split}
\end{equation}

analogamente para \textbf{B} temos,

\begin{equation}
	\frac{1}{4\pi }\oint_{S(V)} da\quad \left[ (\hat{n}\cdot \textbf{B})\textbf{B} - \frac{1}{2}\hat{n} (\textbf{B}\cdot\textbf{B})\right]
\end{equation}

finalmente, conseguimos provar as equações para a conservação de momento linear uma vez que consideramos a densidade de momento linear \textbf{g} e o tensor de estresse de Maxwell ($T_{km}$):

\begin{equation}
	\frac{d}{dt} \textbf{P}_m + \int_V d^3 r \frac{1}{4\pi c} \frac{\partial}{\partial t}\left( \textbf{E} \times \textbf{B}\right) = \frac{d}{dt}\left(\textbf{P}_m + \textbf{P}_c \right)
\end{equation}
 
 com 
 
\begin{equation*}
	\textbf{P}_c =  \int_V d^3 r \quad\textbf{g}, \qquad \textbf{g} \equiv \frac{1}{4\pi c}\left( \textbf{E} \times \textbf{B}\right)
\end{equation*}

portanto, 

\begin{equation}
	\begin{split}
	\frac{d}{dt}\left(\textbf{P}_m + \textbf{P}_c \right) = & \frac{1}{4\pi}\oint_{S(V)} da	\left[ (\hat{n}\cdot \textbf{E})\textbf{E} - \frac{1}{2}\hat{n} (\textbf{E}\cdot\textbf{E}) + (\hat{n}\cdot \textbf{B})\textbf{B} - \frac{1}{2}\hat{n} (\textbf{B}\cdot\textbf{B})\right] \\
	\frac{d}{dt}\left(\textbf{P}_m + \textbf{P}_c \right) = & \oint_{S(V)} da T_{km}n_m
	\end{split}
\end{equation}

com

\begin{equation}
	T_{km} \equiv (\hat{n}\cdot \textbf{E})\textbf{E} - \frac{1}{2}\hat{n} (\textbf{E}\cdot\textbf{E}) +(\hat{n}\cdot \textbf{B})\textbf{B} - \frac{1}{2}\hat{n} (\textbf{B}\cdot\textbf{B}).
\end{equation}

Vale lembrar que, para que tenhamos conservação de momento dentro de um volume, isso acontecerá no caso da integral de superfície do tensor de estresse de maxwell nessa região V for nula, isto é,


\begin{equation}
	\oint_{S(V)} da \quad T_{km}n_m = 0
\end{equation}