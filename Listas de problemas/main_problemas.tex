\documentclass[12pt, a4paper]{article}
\usepackage[T1]{fontenc}
\usepackage[portuguese]{babel}
\usepackage[left=3cm, right=2cm, top=3cm, bottom=2cm]{geometry}
\usepackage{graphicx}
\usepackage{mathtools}
\usepackage{amssymb}
\usepackage{amsthm}
\usepackage{thmtools}
\usepackage{xcolor}
\usepackage{nameref}
\usepackage{babel}
\usepackage{hyperref}
\usepackage{pxfonts}
\usepackage{calrsfs}
\usepackage{amsmath}
\usepackage{fancyhdr}
\usepackage{tcolorbox}



% Contador de parágrafos
\newcounter{paragraphcounter}
\setcounter{paragraphcounter}{0}

% Comando para incrementar o contador de parágrafos
\newcommand{\paragraphnum}{
	\stepcounter{paragraphcounter}
	\theparagraphcounter
}

\pagestyle{fancy}

% Configuração da numeração das equações
\makeatletter
\renewcommand{\theequation}{\theparagraphcounter.\arabic{equation}}
\@addtoreset{equation}{paragraphcounter}
\makeatother

\fancyhead[L]{Eduardo F. Rabelo - 11272697} % Cabeçalho esquerdo
\fancyhead[C]{Problema \theparagraphcounter}     % Cabeçalho central: o capítulo
\fancyhead[R]{SFI5708 - Eletromagnetismo}      % Cabeçalho direito: número da página

%\color{blue}
%\textbf{Problema \paragraphnum: 28/08 }
%	%Insira o enunciado aqui:
%
%\color{black}
%\bigskip\\
%	%Resolução...	
%


\begin{document}

	\color{blue}

\textbf{Problema \paragraphnum: 28/08}
	%Insira o enunciado aqui:
	
	With the same assumptions as in Problem 6.10 discuss the conservation of angular momentum. Show that the differential and integral forms of the conservation law are
	
	\begin{equation*}
		\frac{\partial}{\partial t}(\mathcal{L}_{mech} + \mathcal{L}_{held}) + \mathbf{\nabla} \cdot \overset{\leftrightarrow}{\textbf{M}} = 0
	\end{equation*}
	
	and
	
	\begin{equation*}
		\frac{d}{dt} \int_V d^3x\quad(\mathcal{L}_{mech} + \mathcal{L}_{held}) + \int_S da\quad \hat{\textbf{n}}\cdot \overset{\leftrightarrow}{\textbf{M}} = 0
	\end{equation*}
	
	where the field angular-momentum density is
	
	\begin{equation*}
		\mathcal{L}_{field} = \textbf{x} \times \textbf{g} = \frac{\mu \epsilon}{4\pi c} x \times (\textbf{E} \times \textbf{H})
	\end{equation*}
	
	and the flux of angular momentum is described by the tensor
	
	\begin{equation*}
		\overset{\leftrightarrow}{\textbf{M}} = \overset{\leftrightarrow}{\textbf{T}} \times x
	\end{equation*}
	
	Note: Here we have used the diadic notation for $\textbf{M}_{ij}$ and $\textbf{T}_{ij}$. The double-headed arrow conveys a fairly obvious meaning. For example, $\hat{n} \cdot \overset{\leftrightarrow}{M}$ is a vector whose jth component is $\Sigma_i n_i M_{ij}$. The second-rank $\overset{\leftrightarrow}{M}$ can be written as a third-rank tensor $M_{ij} = T_{ij}x_k - T_{ik}x_j$. But the indices j and k is antisymmetric and so has only three independent elements. Including the index i, $M_ij$ therefore has nine components and can be written as a pseudo tensor of a second rank, as above.

\color{black}
\bigskip

Vamos assumir a conservação de momento linear deduzida em classe e também na lista de exercícios 3:

\begin{equation}
	\frac{d}{dt}\left( \textbf{P}_m + \textbf{P}_c \right)_k = \sum_{m=1}^{3}\oint_{S(V)} T_{km} n_m
	\label{eq1:conservaocaodemomento}
\end{equation}

Uma vez que temos o momento angular definido como o produto vetorial da posição com o momento das massas carregadas definido como

\begin{equation}
	\textbf{L} = \textbf{r} \times \textbf{P}_m
\end{equation}

Assim como no caso do momento linear queremos analisar suas características de conservação. Para ser conservada temos então que visualizar sua variação no tempo dada por


\begin{equation}
	\begin{split}
		\frac{d}{dt}\textbf{L} &= \frac{d}{dt}\left( \textbf{r} \times \textbf{P}_m\right) \\
		& = \frac{d}{dt}\textbf{r} \times \textbf{P}_m + \textbf{r} \times \frac{d}{dt} \textbf{P}_m \\
		& = \textbf{r} \times \frac{d}{dt} \textbf{P}_m
	\end{split}
	\label{eq1:momentoangular}
\end{equation}

Podemos substituir \ref{eq1:conservaocaodemomento} em \ref{eq1:momentoangular} para trabalharmos explicitamente

\begin{equation}
	\begin{split}
		\frac{d}{dt}\textbf{L} & = \textbf{r} \times \frac{d}{dt} \textbf{P}_m \\
		& = \textbf{r} \times\left[ \oint_{S(V)} T_{km} n_m - \frac{d}{dt}\int_V d^3r'\frac{1}{4\pi c} \textbf{E} \times \textbf{B}\right]
		\label{eq1:momentoexplicito}
	\end{split}
\end{equation}

uma vez que

\begin{equation}
	\frac{d}{dt}\textbf{P}_c = \frac{d}{dt}\int_{V_{\infty}} d^3r'\frac{1}{4\pi c} \textbf{E} \times \textbf{B}
\end{equation}

Aplicando o teorema de Gauss na integral de linha em \ref{eq1:momentoexplicito} considerando as propriedades vetoriais apontadas no enunciado do problema chegamos em

\begin{equation}
	\begin{split}
		\frac{d}{dt}\mathbf{L} & = \textbf{r} \times \left[ \int_{V_{\infty}} d^3r' \mathbf{\nabla} \cdot \overset{\leftrightarrow}{\textbf{T}} - \frac{d}{dt}\int_V d^3r'\frac{1}{4\pi c} \textbf{E} \times \textbf{B}\right] \\
		& = \int_{V_{\infty}} d^3r' \quad \textbf{r} \times \left[ \mathbf{\nabla} \cdot \overset{\leftrightarrow}{\textbf{T}} - \frac{d}{dt}\frac{1}{4\pi c} \textbf{E} \times \textbf{B}\right] \\
		& = \int_{V_{\infty}} d^3r'  \quad \left[-\mathbf{\nabla} \cdot \left( \mathbf{r} \times\overset{\leftrightarrow}{\textbf{T}} \right)-\frac{d}{dt} \left( \mathbf{r} \times \mathbf{g} \right) \right] \\
		& = \int_{V_{\infty}} d^3r'  \quad \left[ -\mathbf{\nabla} \cdot \overset{\leftrightarrow}{\textbf{M}} - \frac{d}{dt} \left( \mathbf{r} \times \mathbf{g} \right) \\
		& = \int_{V_{\infty}} d^3r'  \quad \left[ -\mathbf{\nabla} \cdot \overset{\leftrightarrow}{\textbf{M}} - \frac{d}{dt} \mathcal{L}_{field} \right]
	\end{split}
\end{equation}

com 

\begin{equation*}
	\overset{\leftrightarrow}{\textbf{M}} =  \overset{\leftrightarrow}{\textbf{T}} \times \mathbf{r} \qquad \text{e} \qquad \mathbf{g} \equiv \frac{1}{4\pi c} \textbf{E} \times \textbf{B}.
\end{equation*}

Finalmente, podemos demonstrar as equações enunciadas anteriormente. Para isso, vamos propor uma densidade de momento angular no espaço $\mathcal{L}_{mech}$.

\begin{equation}
	\textbf{L} = \int_V d^3r' \mathcal{L}_{mech} \implies \frac{d}{dt}\mathbf{L} = \frac{d}{dt} \int_V d^3r' \mathcal{L}_{mech} 
\end{equation}

logo,

\begin{equation}
	\begin{split}
		& \frac{d}{dt} \int_V d^3r' \mathcal{L}_{mech} = \int_{V_{\infty}} d^3r'  \quad \left[ -\mathbf{\nabla} \cdot \overset{\leftrightarrow}{\textbf{M}} - \frac{d}{dt} \mathcal{L}_{field} \right] \\
		& \frac{d}{dt} \int_V d^3r' \left(\mathcal{L}_{mech} + \mathcal{L}_{field} \right) =- \int_{V_{\infty}} d^3r' \mathbf{\nabla} \cdot \overset{\leftrightarrow}{\textbf{M}} \\
	\end{split}
\end{equation}

assim,

\begin{equation}
	\frac{d}{dt} \oint_V d^3r' \left(\mathcal{L}_{mech} + \mathcal{L}_{field} \right) + \int_{S(V)} da\quad \mathbf{\hat{n}} \cdot \overset{\leftrightarrow}{\textbf{M}} = 0
\end{equation}

\begin{equation}
	\frac{\partial}{\partial t} \left(\mathcal{L}_{mech} + \mathcal{L}_{field} \right) + \mathbf{\nabla} \cdot \overset{\leftrightarrow}{\textbf{M}} = 0
\end{equation}





	
	\newpage

\end{document}