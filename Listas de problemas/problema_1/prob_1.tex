\color{blue}

\textbf{Problema \paragraphnum: 28/08}
	%Insira o enunciado aqui:
	
	With the same assumptions as in Problem 6.10 discuss the conservation of angular momentum. Show that the differential and integral forms of the conservation law are
	
	\begin{equation*}
		\frac{\partial}{\partial t}(\mathcal{L}_{mech} + \mathcal{L}_{held}) + \mathbf{\nabla} \cdot \overset{\leftrightarrow}{\textbf{M}} = 0
	\end{equation*}
	
	and
	
	\begin{equation*}
		\frac{d}{dt} \int_V d^3x\quad(\mathcal{L}_{mech} + \mathcal{L}_{held}) + \int_S da\quad \hat{\textbf{n}}\cdot \overset{\leftrightarrow}{\textbf{M}} = 0
	\end{equation*}
	
	where the field angular-momentum density is
	
	\begin{equation*}
		\mathcal{L}_{field} = \textbf{x} \times \textbf{g} = \frac{\mu \epsilon}{4\pi c} x \times (\textbf{E} \times \textbf{H})
	\end{equation*}
	
	and the flux of angular momentum is described by the tensor
	
	\begin{equation*}
		\overset{\leftrightarrow}{\textbf{M}} = \overset{\leftrightarrow}{\textbf{T}} \times x
	\end{equation*}
	
	Note: Here we have used the diadic notation for $\textbf{M}_{ij}$ and $\textbf{T}_{ij}$. The double-headed arrow conveys a fairly obvious meaning. For example, $\hat{n} \cdot \overset{\leftrightarrow}{M}$ is a vector whose jth component is $\Sigma_i n_i M_{ij}$. The second-rank $\overset{\leftrightarrow}{M}$ can be written as a third-rank tensor $M_{ij} = T_{ij}x_k - T_{ik}x_j$. But the indices j and k is antisymmetric and so has only three independent elements. Including the index i, $M_ij$ therefore has nine components and can be written as a pseudo tensor of a second rank, as above.

\color{black}
\bigskip

Teste
